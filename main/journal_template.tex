\documentclass[12pt]{book}

\usepackage{../main/journal_template_style}
\usepackage{chapterbib}

\title{Research Journal}

\begin{document}
\renewcommand{\printbib}[1]{}% This removes all subfile bibliographies

\maketitle
\tableofcontents

\chapter{Introduction}
This is a template for a mathematics research journal.  It uses the \verb|subfiles| package to split the document up into chapters/topic areas, which allows you to work on each `project' separately without needing to re-compile the entire document. 

\paragraph{Folder Structure}
The folder structure is important.  I keep each project in its own folder: 

\dirtree{%
.1 /.
.2 main.
.3 journal.tex.
.3 journal\_style.sty.
.3 journal\_bib.bib.
.3 figures.
.4 fig1.pdf.
.2 project\_1.
.3 subfile\_1.tex.
.3 figures.
.4 project1fig1.pdf.
.2 project\_2.
.3 subfile\_2.tex.
.3 figures. 
.4 project2fig2.pdf.
}    

\paragraph{Subfile Document Structure}Each project file has the following document class: 
\begin{verbatim}
\documentclass[../main/journal.tex]{subfiles}
\end{verbatim}  Otherwise, when editing a subfile, there is nothing different - work on it like a regular document and compile it separately using your standard compilation workflow (there are no special tricks to get it to compile!)


\chapter{Math}
\subfile{../math/math.tex}

\chapter{Physics}
\subfile{../physics/physics.tex}



\bibliographystyle{plain}
\bibliography{../main/journal_template_bib}



\end{document}